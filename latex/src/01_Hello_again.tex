
% den vorgegebenen header einbinden
\input{Header}

% ein paar Voreinstellungen,
% bitte entsprechend anpassen
\title{Hello again!} % Name der Aufgabe
\author{} % Falls gewünscht, kann man sich als Author eintragen
\renewcommand{\difficulty}{Easy} % Schwierigkeitsgrad der Aufgabe
\renewcommand{\requirements}{Variables and input/output} % Benötigte Voraussetzungen für die Aufgabe
\renewcommand{\aims}{Printing on comand line} % Lernziele der Aufgabe

\begin{document}
% Hier gibt es nichts zu tun
 \maketitle
 \taskinfos
% Ab hier kann man nach Belieben schalten und walten
% Teilaufgaben können mit sections, subsections und subsubsections abgetrennt werden.
% die Numerierung erfolgt automatisch
% \subsection{Ein Teilschritt}
% \subsubsection{Ganz genaue Anweisungen}
\ \\\ \\
\begin{itemize}
	\item Write a program that prints the String "Hello World!"\ on the command line, using only placeholders in the format string.
	\item \textbf{Experts:} In leetspeek you say "H3110 W0$|$21d!". Use actual numbers.
\end{itemize}

\end{document}
