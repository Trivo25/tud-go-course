\input{course_definitions}	% nothing to do here
\usepackage[english]{babel}

\usepackage[utf8]{inputenc}

\title{Go}
\subtitle{ Fast, reliable, and efficient software at scale}
\date{\today}
\author{Moritz Schulz, Florian Kluge}
\institute{Wintersemester 2021/22}

\lstset{
  language = Go,
  showspaces = false,
  showtabs = false,
  showstringspaces = false,
  escapechar = @,
  belowskip=-1.5em
}

\def\ContinueLineNumber{\lstset{firstnumber=last}}
\def\StartLineAt#1{\lstset{firstnumber=#1}}
\let\numberLineAt\StartLineAt

\makeatletter
 % TODO modify this if you have not already done so


% meta-information
\usepackage{tikz}
\usepackage{hyperref}
\hypersetup{
	colorlinks=true,
	linkcolor=darkgray,
	urlcolor=blue,
}

\begin{document}

\maketitle

\begin{frame}{Übersicht}
	\setbeamertemplate{section in toc}[sections numbered]
	\tableofcontents[hideallsubsections]
\end{frame}


% ################# ORGANISATION #################

\section{Organisation}

\subsection{Kontaktmöglichkeiten}
\begin{frame}{Page-Title}
	\begin{center}
		center
	\end{center}
\end{frame}

\subsection{Ablauf}
\begin{frame}{Page-Title}
	Sample
\end{frame}

\subsection{Ziel}
\begin{frame}{Page-Title}
	Sample
\end{frame}


% ################# Wieso Go? #################

\section{Wieso Go?}
\begin{frame}{Page-Title}
	\lstinputlisting[language=go]{../code_samples/01_hello_world.go}
\end{frame}


\subsection{Was eigentlich ist Go?}
\begin{frame}{Page-Title}
	Sample
\end{frame}


\subsection{Anwendungen und Beispiele}
\begin{frame}{Page-Title}
	Sample
\end{frame}


% ################# nützliche Links #################

\section{nützliche Links}

\begin{frame}{Literatur}
	Bodner, Jon: Learning Go, 2021. ISBN-10: 1492077216
\end{frame}



% ################# Entwicklungsumgebungen #################

\section{Entwicklungsumgebungen}


\subsection{Playground im Browser}
\begin{frame}{Page-Title}
	Sample
\end{frame}

\subsection{Lokal}
\begin{frame}{Page-Title}
	Sample
\end{frame}

% nothing to do from here on
\end{document}
