\input{course_definitions}	% nothing to do here
\usepackage[english]{babel}

\usepackage[utf8]{inputenc}

\title{Go}
\subtitle{ Fast, reliable, and efficient software at scale}
\date{\today}
\author{Moritz Schulz, Florian Kluge}
\institute{Wintersemester 2021/22}

\lstset{
  language = Go,
  showspaces = false,
  showtabs = false,
  showstringspaces = false,
  escapechar = @,
  belowskip=-1.5em
}

\def\ContinueLineNumber{\lstset{firstnumber=last}}
\def\StartLineAt#1{\lstset{firstnumber=#1}}
\let\numberLineAt\StartLineAt

\makeatletter
 % TODO modify this if you have not already done so


% meta-information
\usepackage{tikz}
\usepackage{hyperref}
\hypersetup{
	colorlinks=true,
	linkcolor=darkgray,
	urlcolor=blue,
}


\title{Go}
\subtitle{Fast, reliable, and efficient software at scale}
\date{\today}
\author{Moritz Schulz, Florian Kluge}
\institute{Florian.Kluge@mailbox.tu-dresden.de, Florian.Kluge@mailbox.tu-dresden.de}
% \titlegraphic{\hfill\includegraphics[height=1.5cm]{logo.pdf}}

\begin{document}
\maketitle


\begin{frame}{Übersicht}
	\setbeamertemplate{section in toc}[sections numbered]
	\tableofcontents[hideallsubsections]
\end{frame}


% ################# ORGANISATION #################

\section{Organisation}

\subsection{Wer sind wir?}
\begin{frame}{Wer sind wir?}

\end{frame}

\subsection{Was machen wir hier?}
\begin{frame}{Was machen wir hier?}
	Sample
\end{frame}


% ################# Wieso Go? #################

\section{Warum Go?}
\begin{frame}{Warum Go?}
	Warum Go?
\end{frame}


\subsection{Vorteile}
\begin{frame}{Vorteile}
	Sample
\end{frame}


\subsection{Anwendungen und Beispiele}
\begin{frame}{Anwendungen und Beispiele}
	Sample
\end{frame}


% ################# Time to get started! ..almost #################

\section{Time to get started! ..almost}

\begin{frame}{Time to get started! ..almost}
	Internet?
\end{frame}



% ################# Los geht's! #################

\section{Los geht's!}


\subsection{Playground im Browser}
\begin{frame}{Spaß im Browser}
	Sample
\end{frame}


\subsection{Was ist da jetzt eigentlich passiert?}
\begin{frame}{Was ist da jetzt eigentlich passiert?}
	Was ist da jetzt eigentlich passiert?
\end{frame}


\subsection{Eure erste Aufgabe}
\begin{frame}{Eure erste Aufgabe}
	..
\end{frame}


% ################# Abschluss #################

\section{Das war's auch schon (für Heut)}


\subsection{}
\begin{frame}{Informationen und Links zum Kurs}
	QR Code
\end{frame}

% nothing to do from here on
\end{document}
