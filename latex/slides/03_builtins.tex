% The Slide Definitions
%% Nothing to modify here.
%% make sure to include this before anything else
% \renewcommand*\oldstylenums[1]{{\firaoldstyle #1}}
\documentclass[10pt,aspectratio=169]{beamer}
\usetheme{metropolis}
\usenavigationsymbolstemplate{}
\usepackage{tgcursor}
\usepackage{appendixnumberbeamer}
\usepackage{xcolor}
\usepackage[scale=2]{ccicons}
\usepackage[sfdefault]{FiraSans} %% option 'sfdefault' activates Fira Sans as the default text font
\usepackage[T1]{fontenc}
\usepackage{pgfplots}
\usepgfplotslibrary{dateplot}

\usepackage{xspace}
\newcommand{\themename}{\textbf{\textsc{metropolis}}\xspace}
% packages
\usepackage{color}
\usepackage{listings}
\usepackage{multicol}

% color definitions
\definecolor{mygreen}{rgb}{0,0.6,0}
\definecolor{mygray}{rgb}{0.5,0.5,0.5}
\definecolor{mymauve}{rgb}{0.58,0,0.82}

% re-format the title frame page
\makeatletter
\def\supertitle#1{\gdef\@supertitle{#1}}%
% \setbeamertemplate{title page}
% {
%     \vbox{}
%     \vfill
%     \begin{centering}
%         \begin{beamercolorbox}[sep=8pt,center]{title}
%             \usebeamerfont{supertitle}\@supertitle
%         \end{beamercolorbox}
%         \begin{beamercolorbox}[sep=8pt,center]{title}
%             \usebeamerfont{title}\inserttitle\par%
%             \ifx\insertsubtitle\@empty%
%             \else%
%                 \vskip0.25em%
%                 {\usebeamerfont{subtitle}\usebeamercolor[fg]{subtitle}\insertsubtitle\par}%
%             \fi%
%         \end{beamercolorbox}%
%         \vskip1em\par
%         \begin{beamercolorbox}[sep=8pt,center]{author}
%             \usebeamerfont{author}\insertauthor
%         \end{beamercolorbox}
%         \begin{beamercolorbox}[sep=8pt,center]{institute}
%             \usebeamerfont{institute}\insertinstitute
%         \end{beamercolorbox}
%         \begin{beamercolorbox}[sep=8pt,center]{date}
%             \usebeamerfont{date}\insertdate
%         \end{beamercolorbox}\vskip0.5em
%         {\usebeamercolor[fg]{titlegraphic}\inserttitlegraphic\par}
%     \end{centering}
%     \vfill
% }
\makeatother

% insert frame number
\expandafter\def\expandafter\insertshorttitle\expandafter{%
    \insertshorttitle\hfill%
    \insertframenumber\,/\,\inserttotalframenumber}

% preset-listing options
\lstset{
    backgroundcolor=\color{white},
    % choose the background color;
    % you must add \usepackage{color} or \usepackage{xcolor}
    basicstyle=\footnotesize,
    % the size of the fonts that are used for the code
    breakatwhitespace=false,
    % sets if automatic breaks should only happen at whitespace
    breaklines=true,                 % sets automatic line breaking
    captionpos=b,                    % sets the caption-position to bottom
    commentstyle=\color{mygreen},    % comment style
    % deletekeywords={...},
    % if you want to delete keywords from the given language
    extendedchars=true,
    % lets you use non-ASCII characters;
    % for 8-bits encodings only, does not work with UTF-8
    frame=single,                    % adds a frame around the code
    keepspaces=true,
    % keeps spaces in text,
    % useful for keeping indentation of code
    % (possibly needs columns=flexible)
    keywordstyle=\color{blue},       % keyword style
    % morekeywords={*,...},
    % if you want to add more keywords to the set
    numbers=left,
    % where to put the line-numbers; possible values are (none, left, right)
    numbersep=5pt,
    % how far the line-numbers are from the code
    numberstyle=\tiny\color{mygray},
    % the style that is used for the line-numbers
    rulecolor=\color{black},
    % if not set, the frame-color may be changed on line-breaks
    % within not-black text (e.g. comments (green here))
    stepnumber=1,
    % the step between two line-numbers.
    % If it's 1, each line will be numbered
    stringstyle=\color{mymauve},     % string literal style
    tabsize=4,                       % sets default tabsize to 4 spaces
    title=\lstname
    % show the filename of files included with \lstinputlisting;
    % also try caption instead of title
}

% macro for code inclusion
\newcommand{\includecode}[2][c]{
    \lstinputlisting[caption=#2, style=custom#1]{#2}
}


% Author and Course information
\usepackage[german]{babel}

\usepackage[utf8]{inputenc}

\title{Go}
\subtitle{ Fast, reliable, and efficient software at scale}
\date{\today}
\author{Moritz Schulz, Florian Kluge}
\institute{
  Mail_1\newline
  Florian.Kluge@mailbox.tu-dresden.de
}

\lstset{
  language = Go,
  showspaces = false,
  showtabs = false,
  showstringspaces = false,
  escapechar = @,
  belowskip=-1.5em
}

\def\ContinueLineNumber{\lstset{firstnumber=last}}
\def\StartLineAt#1{\lstset{firstnumber=#1}}
\let\numberLineAt\StartLineAt

\makeatletter


% Presentation title
\title{Builtin Datenstrukturen}
\date{\today}


\begin{document}

\maketitle

\begin{frame}{Gliederung}
	\begin{multicols}{2}
		\setbeamertemplate{section in toc}[sections numbered]
		\tableofcontents
	\end{multicols}
\end{frame}


% #############################################################################
% ------------------------------- 1. Exceptions -------------------------------
% #############################################################################
\section{Exceptions}

\begin{frame}{Exception Handling}
	\begin{itemize}
		\item Alle Exceptions erben von \alert{\texttt{Exception}}
		\item Catching mit try/except
		\item \alert{\texttt{finally}} um Code auszuführen, der \textit{unbedingt} laufen muss, egal ob eine Exception vorliegt oder nicht
	\end{itemize}
\end{frame}

\begin{frame}{Exception Handling - Beispiel}
    \lstinputlisting{resources/03_builtins/exceptions.py}
\end{frame}


% #############################################################################
% -------------------------------- 2. Booleans --------------------------------
% #############################################################################
\section{Booleans}

\begin{frame}{Boolsche Werte}
	\begin{itemize}
		\item \textit{type} ist \alert{\texttt{bool}}
		\item Mögliche Werte: \texttt{True} oder \texttt{False}
		\item Operationen sind \textit{und}, \textit{oder}, \textit{nicht} (\texttt{and, or, not})
	\end{itemize}
\end{frame}


% #############################################################################
% --------------------------------- 3. Lists ----------------------------------
% #############################################################################
\section{Lists}

\begin{frame}{list}
	\begin{itemize}
		\item enthält variable Anzahl von Objekten
		\item eine Liste kann beliebig viele verschiedene Datentypen enthalten (z.B. \texttt{bool} und \texttt{list})
		\item Auch Listen können in Listen gespeichert werden!
		\item Listenobjekte haben eine feste Reihenfolge (\textit{first in, last out})
		\item optimiert für einseitige Benutzung wie z.B. Queue (\alert{\texttt{append}} und \alert{\texttt{pop}})
	\end{itemize}
\end{frame}

\begin{frame}{list - Beispiel}
	\lstinputlisting{resources/03_builtins/list.py}
\end{frame}


% #############################################################################
% --------------------------------- 4. Tuples ---------------------------------
% #############################################################################
\section{Tuples}

\begin{frame}{tuple}
	\begin{itemize}
		\item Gruppiert Daten
		\item kann nicht mehr verändert werden, sobald es erstellt wurde
		\item Funktionen mit mehreren Rückgabewerten geben ein Tupel zurück
	\end{itemize}
\end{frame}

\begin{frame}{tuple - Beispiel}
	\lstinputlisting{resources/03_builtins/tuple.py}
\end{frame}


% #############################################################################
% --------------------------------- 5. Dicts ----------------------------------
% #############################################################################
\section{Dicts}

\begin{frame}{dict}
	\begin{itemize}
		\item einfache Hashmap
		\item ungeordnet
		\item jeder hashbare Typ kann ein Key sein
		\item jedem Key ist dann ein Value zugeordnet
	\end{itemize}
\end{frame}

\begin{frame}{dict - Beispiel}
    \lstinputlisting[lastline=10]{resources/03_builtins/dict.py}
\end{frame}

\begin{frame}{dict - Beispiel}
    \lstinputlisting[firstline=11]{resources/03_builtins/dict.py}
\end{frame}


% #############################################################################
% ----------------------------- 6. Set/Frozenset ------------------------------
% #############################################################################
\section{Set/Frozenset}

\begin{frame}{set/frozenset}
	\begin{itemize}
		\item kann nur hashbare Einträge enthalten
		\item \texttt{set} selbst ist nicht hashbar
		\item \texttt{frozensets} sind hashbar, jedoch nicht mehr veränderbar
		\item enthält jedes Element nur einmal
		\item schnellere Überprüfung mit \alert{\texttt{in}} (prüft, ob Element enthalten ist)
		\item Mögliche Operationen: \alert{\texttt{superset()}}, \alert{\texttt{subset()}}, \alert{\texttt{isdisjoint()}}, \alert{\texttt{difference()}}, \alert{\texttt{<}}, \alert{\texttt{>}}, \alert{\texttt{disjoint()}}, \alert{\texttt{-}}
		\item ungeordnet
		\item (frozen)sets können frozensets enthalten (da sie einen festen Hashwert haben)
	\end{itemize}
\end{frame}

\begin{frame}{set/frozenset - Beispiel}
	\lstinputlisting{resources/03_builtins/set.py}
\end{frame}


% #############################################################################
% ------------------------------- 7. Iterations -------------------------------
% #############################################################################
\section{Iterations}

\begin{frame}{Iteration}
	\begin{itemize}
		\item nur foreach
		\item für Iterationen über Integer gibt es \\ \hspace{0.5cm} \texttt{range([start], stop, step=1)}
		\item um Iteratoren zu kombinieren kann man \\ \hspace{0.5cm} \texttt{zip(iterator\_1, iterator\_2, ..., iterator\_n)} verwenden
		\item alles mit einer \alert{\texttt{\_\_iter\_\_}} Methode ist iterierbar
		% TODO: stateful Iterator erklaeren oder entfernen
		\item \texttt{iter(iterable)} konstruiert einen \textit{stateful iterator}
	\end{itemize}
\end{frame}

\begin{frame}{Iteration - Beispiel}
	\lstinputlisting[lastline=15]{resources/03_builtins/iterate.py}
\end{frame}

\begin{frame}{Iteration - Beispiel}
	\lstinputlisting[firstline=18]{resources/03_builtins/iterate.py}
\end{frame}


% #############################################################################
% ------------------------------- 8. Unpacking --------------------------------
% #############################################################################
\section{Unpacking}

\begin{frame}{Unpacking}
	\begin{itemize}
		\item einfaches Auflösen von Listen und Tupeln in einzelne Variablen
		\item nützlich in \alert{\texttt{for}}-Schleifen
	\end{itemize}
\end{frame}

\begin{frame}{Unpacking - Beispiel}
	\lstinputlisting{resources/03_builtins/unpacking.py}
\end{frame}


% #############################################################################
% ----------------------------- 9. File Handling ------------------------------
% #############################################################################
\section{File Handling}

\begin{frame}{File Handling}
	\begin{itemize}
		\item Dateien können mit \alert{\texttt{open(filename, mode="r")}} geöffnet werden
		\item \textit{File Handler} sind Iteratoren über die Zeilen einer Datei
		\item \textbf{Wichtig:} File Handler müssen auch wieder geschlossen werden
		\item \texttt{r} steht für Lesezugriff,  \texttt{w} für Schreibzugriff
	\end{itemize}
	\textbf{Beachte:} Wird eine Datei mit Schreibzugriff geöffnet, wird sie geleert! Also wichtige Inhalte vorher auslesen.
\end{frame}

\begin{frame}{File Handling - Beispiel}
	\lstinputlisting{resources/03_builtins/file.py}
\end{frame}


% #############################################################################
% ---------------------------- 10. Context Manager ----------------------------
% #############################################################################
\section{Context Manager}

\begin{frame}{Context Manager}
	\begin{itemize}
		\item Aufruf mit \alert{\texttt{with}}
		\item kann jedes Objekt sein, welches eine \texttt{\_\_enter\_\_} und \texttt{\_\_exit\_\_} Methode hat
		\item praktisch beim \textit{File Handling}
	\end{itemize}
\end{frame}

\begin{frame}{Context Manager}
	\lstinputlisting{resources/03_builtins/cm.py}
\end{frame}


\end{document}
