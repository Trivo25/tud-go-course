% The Slide Definitions
%% Nothing to modify here.
%% make sure to include this before anything else
% \renewcommand*\oldstylenums[1]{{\firaoldstyle #1}}
\documentclass[10pt,aspectratio=169]{beamer}
\usetheme{metropolis}
\usenavigationsymbolstemplate{}
\usepackage{tgcursor}
\usepackage{appendixnumberbeamer}
\usepackage{xcolor}
\usepackage[scale=2]{ccicons}
\usepackage[sfdefault]{FiraSans} %% option 'sfdefault' activates Fira Sans as the default text font
\usepackage[T1]{fontenc}
\usepackage{pgfplots}
\usepgfplotslibrary{dateplot}

\usepackage{xspace}
\newcommand{\themename}{\textbf{\textsc{metropolis}}\xspace}
% packages
\usepackage{color}
\usepackage{listings}
\usepackage{multicol}

% color definitions
\definecolor{mygreen}{rgb}{0,0.6,0}
\definecolor{mygray}{rgb}{0.5,0.5,0.5}
\definecolor{mymauve}{rgb}{0.58,0,0.82}

% re-format the title frame page
\makeatletter
\def\supertitle#1{\gdef\@supertitle{#1}}%
% \setbeamertemplate{title page}
% {
%     \vbox{}
%     \vfill
%     \begin{centering}
%         \begin{beamercolorbox}[sep=8pt,center]{title}
%             \usebeamerfont{supertitle}\@supertitle
%         \end{beamercolorbox}
%         \begin{beamercolorbox}[sep=8pt,center]{title}
%             \usebeamerfont{title}\inserttitle\par%
%             \ifx\insertsubtitle\@empty%
%             \else%
%                 \vskip0.25em%
%                 {\usebeamerfont{subtitle}\usebeamercolor[fg]{subtitle}\insertsubtitle\par}%
%             \fi%
%         \end{beamercolorbox}%
%         \vskip1em\par
%         \begin{beamercolorbox}[sep=8pt,center]{author}
%             \usebeamerfont{author}\insertauthor
%         \end{beamercolorbox}
%         \begin{beamercolorbox}[sep=8pt,center]{institute}
%             \usebeamerfont{institute}\insertinstitute
%         \end{beamercolorbox}
%         \begin{beamercolorbox}[sep=8pt,center]{date}
%             \usebeamerfont{date}\insertdate
%         \end{beamercolorbox}\vskip0.5em
%         {\usebeamercolor[fg]{titlegraphic}\inserttitlegraphic\par}
%     \end{centering}
%     \vfill
% }
\makeatother

% insert frame number
\expandafter\def\expandafter\insertshorttitle\expandafter{%
    \insertshorttitle\hfill%
    \insertframenumber\,/\,\inserttotalframenumber}

% preset-listing options
\lstset{
    backgroundcolor=\color{white},
    % choose the background color;
    % you must add \usepackage{color} or \usepackage{xcolor}
    basicstyle=\footnotesize,
    % the size of the fonts that are used for the code
    breakatwhitespace=false,
    % sets if automatic breaks should only happen at whitespace
    breaklines=true,                 % sets automatic line breaking
    captionpos=b,                    % sets the caption-position to bottom
    commentstyle=\color{mygreen},    % comment style
    % deletekeywords={...},
    % if you want to delete keywords from the given language
    extendedchars=true,
    % lets you use non-ASCII characters;
    % for 8-bits encodings only, does not work with UTF-8
    frame=single,                    % adds a frame around the code
    keepspaces=true,
    % keeps spaces in text,
    % useful for keeping indentation of code
    % (possibly needs columns=flexible)
    keywordstyle=\color{blue},       % keyword style
    % morekeywords={*,...},
    % if you want to add more keywords to the set
    numbers=left,
    % where to put the line-numbers; possible values are (none, left, right)
    numbersep=5pt,
    % how far the line-numbers are from the code
    numberstyle=\tiny\color{mygray},
    % the style that is used for the line-numbers
    rulecolor=\color{black},
    % if not set, the frame-color may be changed on line-breaks
    % within not-black text (e.g. comments (green here))
    stepnumber=1,
    % the step between two line-numbers.
    % If it's 1, each line will be numbered
    stringstyle=\color{mymauve},     % string literal style
    tabsize=4,                       % sets default tabsize to 4 spaces
    title=\lstname
    % show the filename of files included with \lstinputlisting;
    % also try caption instead of title
}

% macro for code inclusion
\newcommand{\includecode}[2][c]{
    \lstinputlisting[caption=#2, style=custom#1]{#2}
}


% Author and Course information
\usepackage[german]{babel}

\usepackage[utf8]{inputenc}

\title{Go}
\subtitle{ Fast, reliable, and efficient software at scale}
\date{\today}
\author{Moritz Schulz, Florian Kluge}
\institute{
  Mail_1\newline
  Florian.Kluge@mailbox.tu-dresden.de
}

\lstset{
  language = Go,
  showspaces = false,
  showtabs = false,
  showstringspaces = false,
  escapechar = @,
  belowskip=-1.5em
}

\def\ContinueLineNumber{\lstset{firstnumber=last}}
\def\StartLineAt#1{\lstset{firstnumber=#1}}
\let\numberLineAt\StartLineAt

\makeatletter


% Presentation title
\title{\texttt{ncurses} in Python}
\date{\today}


\begin{document}

\maketitle

\begin{frame}{Gliederung}
	\setbeamertemplate{section in toc}[sections numbered]
	\tableofcontents
\end{frame}



\section{Das Modul \texttt{curses}}

\begin{frame}{Das Modul \texttt{curses}}
	\texttt{\alert{curses}} bietet ein Interface für die \glqq{}curses\grqq{}-Bibliothek, den de-facto Standard für fortgeschrittenes Terminal Handling.
\end{frame}

\begin{frame}{Das Modul \texttt{curses}}
	\begin{itemize}
		\item curses ist vor allem in Unix-Umgebungen weit verbreitet
		\item dieses Modul wurde auf die API von \emph{ncurses}, einer Open-Source Library  für curses, zugeschnitten und funktioniert nur unter Linux und BSD Varianten von Unix
	\end{itemize}
\end{frame}

\begin{frame}{Mitgelieferte Submodule}
	\begin{description}
		\item [\texttt{curses.ascii}] zur Arbeit mit ASCII-kodierten Zeichen
		\item [\texttt{curses.panel}] Support für Panels, also mehrere Fenster übereinander (Fenster mit Ebenen)
		\item [\texttt{curses.textpad}] Widget zum editieren von Text mit \emph{emacs}-artigen Key-Bindings
	\end{description}	
\end{frame}


\section{Erstellen eines simplen Fensters}

\subsection{Grundlegende Funktionen}

\begin{frame}{Grundlegende Funktionen}
	Zum Erstellen eines \texttt{curses}-Fensters benutzt man den Befehl
	
	\lstinputlisting[lastline=1]{resources/12_curses/functions.py}
	
	\begin{itemize}
		\item instanziiert die Bibliothek, gibt ein \texttt{WindowObject} zurück, welches den gesamten Screen repräsentiert
		\item Fehler beim Terminal öffnen können zum Beenden des Interpreters führen!
	\end{itemize}
\end{frame}

\begin{frame}{Grundlegende Funktionen}
	\lstinputlisting[firstline=3, lastline=3]{resources/12_curses/functions.py}
	
	\begin{itemize}
		\item deinitialisiert Bibliothek
		\item führt Terminal in normalen Zustand zurück
	\end{itemize}
\end{frame}

\begin{frame}
	\texttt{curses} liefert einen eigenen Wrapper zum Starten und Beenden:
	
	\lstinputlisting[firstline=5, lastline=5]{resources/12_curses/functions.py}
	
	\begin{itemize}
		\item setzt nützliche Standards, initialisiert curses und ruft \emph{func} auf
		\item Wrapper fängt Exceptions, stellt Terminal wieder her und wirft Exception erneut
		\item \emph{func} muss \alert{\texttt{stdscr}} als erstes Argument akzeptieren (für das \texttt{WindowObject})
	\end{itemize}
\end{frame}


\subsection{Darstellen von Symbolen}

\begin{frame}{Darstellen von Symbolen}
	Zur Darstellung einzelner Symbole verwendet man \texttt{addch}:
	
	\lstinputlisting[firstline=7, lastline=8]{resources/12_curses/functions.py}
	
	\begin{description}
		\item [\texttt{ch}] zu schreibendes ASCII-Zeichen (im ASCII-Format!) \\ Zum konvertieren ggf. auf \texttt{ord()} zurück greifen.
		\item [\texttt{y, x}] Koordinaten, an die das Zeichen geschrieben wird. Beachte invertierte Reihenfolge!
		\item [\texttt{attr}] zu setzende Attribute
	\end{description}
\end{frame}

\begin{frame}{Darstellen von Symbolen}
	Zum Darstellen eines Strings existiert die Funktion \texttt{addstr()}:
	
	\lstinputlisting[firstline=10, lastline=11]{resources/12_curses/functions.py}
	
	\begin{itemize}
		\item fügt ab \texttt{(y, x)} den gegebenen String ein
		\item überschreibt alle vorher vorhandenen Zeichen
	\end{itemize}
\end{frame}


\begin{frame}{Zeichen entfernen}
	\lstinputlisting[firstline=13, lastline=15]{resources/12_curses/functions.py}
	
	\begin{itemize}
		\item Funktionen zum löschen einzelner Zeichen, einer ganzen Zeile oder dem gesamten Fenster
	\end{itemize}
\end{frame}


\subsection{Allgemeine Hinweise}

\begin{frame}{Allgemeine Hinweise}
	\begin{itemize}
		\item Werden bei Funktionen \texttt{x} und \texttt{y} weg gelassen und sind optional, wird die Funktion an der aktuellen Cursorstelle ausgeführt.
		\item Beachte die invertierte Zählweise!
		\begin{itemize}
			\item Punkt (0, 0) liegt oben links in der Ecke vom Terminal
			\item x zählt von da nach rechts aufwärts
			\item y zählt von da nach unten aufwärts
		\end{itemize}
	\end{itemize}
\end{frame}

\section{Arbeiten mit \texttt{curses.textpad}}
\begin{frame}{\texttt{rectangle}}
	\lstinputlisting[firstline=17, lastline=17]{resources/12_curses/functions.py}
	\begin{itemize}
		\item erzeugt im WindowObject \texttt{win} ein Rechteck
		\item Anfangspunkt ist (\texttt{ulx}, \texttt{uly}) (ul $\hat=$ upper left corner)
		\item Endpunkt ist (\texttt{lrx}, \texttt{lry}) (lr $\hat=$ lower right corner)
	\end{itemize}
\end{frame}

\begin{frame}{\texttt{Textbox}}
	\lstinputlisting[firstline=19, lastline=23]{resources/12_curses/functions.py}
	\begin{itemize}
		\item Eine \texttt{Textbox} wird mit einem WindowObject erstellt und hat diese drei Funktionen
		\item \texttt{edit()} \glqq{}Editormodus\grqq{} wobei \texttt{validator} eine Funktion ist
		\item \texttt{do\_command()} f\"uhrt einen Befehl aus \href{http://bit.ly/29sJ8QJ}{(Link hier)}
		\item \texttt{gather()} gibt den Inhalt der Textbox zur\"uck
	\end{itemize}
\end{frame}


\end{document}
