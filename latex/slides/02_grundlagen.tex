% The Slide Definitions
%% Nothing to modify here.
%% make sure to include this before anything else
% \renewcommand*\oldstylenums[1]{{\firaoldstyle #1}}
\documentclass[10pt,aspectratio=169]{beamer}
\usetheme{metropolis}
\usenavigationsymbolstemplate{}
\usepackage{tgcursor}
\usepackage{appendixnumberbeamer}
\usepackage{xcolor}
\usepackage[scale=2]{ccicons}
\usepackage[sfdefault]{FiraSans} %% option 'sfdefault' activates Fira Sans as the default text font
\usepackage[T1]{fontenc}
\usepackage{pgfplots}
\usepgfplotslibrary{dateplot}

\usepackage{xspace}
\newcommand{\themename}{\textbf{\textsc{metropolis}}\xspace}
% packages
\usepackage{color}
\usepackage{listings}
\usepackage{multicol}

% color definitions
\definecolor{mygreen}{rgb}{0,0.6,0}
\definecolor{mygray}{rgb}{0.5,0.5,0.5}
\definecolor{mymauve}{rgb}{0.58,0,0.82}

% re-format the title frame page
\makeatletter
\def\supertitle#1{\gdef\@supertitle{#1}}%
% \setbeamertemplate{title page}
% {
%     \vbox{}
%     \vfill
%     \begin{centering}
%         \begin{beamercolorbox}[sep=8pt,center]{title}
%             \usebeamerfont{supertitle}\@supertitle
%         \end{beamercolorbox}
%         \begin{beamercolorbox}[sep=8pt,center]{title}
%             \usebeamerfont{title}\inserttitle\par%
%             \ifx\insertsubtitle\@empty%
%             \else%
%                 \vskip0.25em%
%                 {\usebeamerfont{subtitle}\usebeamercolor[fg]{subtitle}\insertsubtitle\par}%
%             \fi%
%         \end{beamercolorbox}%
%         \vskip1em\par
%         \begin{beamercolorbox}[sep=8pt,center]{author}
%             \usebeamerfont{author}\insertauthor
%         \end{beamercolorbox}
%         \begin{beamercolorbox}[sep=8pt,center]{institute}
%             \usebeamerfont{institute}\insertinstitute
%         \end{beamercolorbox}
%         \begin{beamercolorbox}[sep=8pt,center]{date}
%             \usebeamerfont{date}\insertdate
%         \end{beamercolorbox}\vskip0.5em
%         {\usebeamercolor[fg]{titlegraphic}\inserttitlegraphic\par}
%     \end{centering}
%     \vfill
% }
\makeatother

% insert frame number
\expandafter\def\expandafter\insertshorttitle\expandafter{%
    \insertshorttitle\hfill%
    \insertframenumber\,/\,\inserttotalframenumber}

% preset-listing options
\lstset{
    backgroundcolor=\color{white},
    % choose the background color;
    % you must add \usepackage{color} or \usepackage{xcolor}
    basicstyle=\footnotesize,
    % the size of the fonts that are used for the code
    breakatwhitespace=false,
    % sets if automatic breaks should only happen at whitespace
    breaklines=true,                 % sets automatic line breaking
    captionpos=b,                    % sets the caption-position to bottom
    commentstyle=\color{mygreen},    % comment style
    % deletekeywords={...},
    % if you want to delete keywords from the given language
    extendedchars=true,
    % lets you use non-ASCII characters;
    % for 8-bits encodings only, does not work with UTF-8
    frame=single,                    % adds a frame around the code
    keepspaces=true,
    % keeps spaces in text,
    % useful for keeping indentation of code
    % (possibly needs columns=flexible)
    keywordstyle=\color{blue},       % keyword style
    % morekeywords={*,...},
    % if you want to add more keywords to the set
    numbers=left,
    % where to put the line-numbers; possible values are (none, left, right)
    numbersep=5pt,
    % how far the line-numbers are from the code
    numberstyle=\tiny\color{mygray},
    % the style that is used for the line-numbers
    rulecolor=\color{black},
    % if not set, the frame-color may be changed on line-breaks
    % within not-black text (e.g. comments (green here))
    stepnumber=1,
    % the step between two line-numbers.
    % If it's 1, each line will be numbered
    stringstyle=\color{mymauve},     % string literal style
    tabsize=4,                       % sets default tabsize to 4 spaces
    title=\lstname
    % show the filename of files included with \lstinputlisting;
    % also try caption instead of title
}

% macro for code inclusion
\newcommand{\includecode}[2][c]{
    \lstinputlisting[caption=#2, style=custom#1]{#2}
}


% Author and Course information
\usepackage[german]{babel}

\usepackage[utf8]{inputenc}

\title{Go}
\subtitle{ Fast, reliable, and efficient software at scale}
\date{\today}
\author{Moritz Schulz, Florian Kluge}
\institute{
  Mail_1\newline
  Florian.Kluge@mailbox.tu-dresden.de
}

\lstset{
  language = Go,
  showspaces = false,
  showtabs = false,
  showstringspaces = false,
  escapechar = @,
  belowskip=-1.5em
}

\def\ContinueLineNumber{\lstset{firstnumber=last}}
\def\StartLineAt#1{\lstset{firstnumber=#1}}
\let\numberLineAt\StartLineAt

\makeatletter


% Presentation title
\title{Grundlagen}
\date{\today}


\begin{document}

\maketitle

\begin{frame}{Gliederung}
	\begin{multicols}{2}
		\setbeamertemplate{section in toc}[sections numbered]
		\tableofcontents
	\end{multicols}
\end{frame}



\section{Scriptcharakter}
\begin{frame}[fragile]{Scriptcharakter}
	\begin{itemize}
		\item Beim Ausführen oder Importieren wird der Code im obersten Level des Moduls (der .py Datei) ausgeführt
		\item Funktionen, Klassen und globale Variablen werden üblicherweise auf dem obersten Level definiert
		\item Imports anderer Python Module werden auch hier ausgeführt
	\end{itemize}
\end{frame}

\begin{frame}[fragile]{Scriptcharakter}
	\lstinputlisting{resources/02_grundlagen/boilerplate.py} 
	Soll das entsprechende Modul ausführbar sein und nicht nur als Bibliothek dienen, definiert man üblicherweise eine \alert{main}-Funktion.\\
	Zus\"atzlich f\"ugt man am Ende des Moduls die Boilerplate ein.\\
\end{frame}

\begin{frame}{Scriptcharakter}
	\begin{itemize}
		\item Python Code wird nicht kompiliert, sondern beim importieren in Python Bytecode übersetzt
		\item Bytecode wird auf einer VM ausgeführt
		\item Kein Memory Management nötig, alles sind Referenzen
		\item Syntaxerror wird beim Importieren geworfen
		\item Andere Fehler findet man erst, wenn die betreffende Zeile ausgeführt wird.
	\end{itemize}
\end{frame}



\section{Programmierparadigmen}
\subsection{Imperatives Programmieren}
\begin{frame}[fragile]{Programmierpradigmen}
	\begin{itemize}
		\item Python ist vor allem eine imperative und objektorientierte Sprache
		\item reine Funktionen und Variablen können auf oberster Ebene definiert werden
		\item Variablen, Klassen und Funktionen sind ab der Ebene sichtbar, in der sie eingeführt werden
	\end{itemize}
\end{frame}

\subsection{Das Scoping Problem}
\begin{frame}[fragile]{Das Problem mit dem Scope}
	\lstinputlisting{resources/02_grundlagen/scope.py}
\end{frame}

\begin{frame}{Das Problem mit dem Scope}
	\textbf{\alert{Das Problem:}} \\[.25cm]
	Variablen sind zwar nach innen sichtbar, werden aber beim Reassignment innerhalb der Funktion neu angelegt und verschwinden so aus dem Scope.
\end{frame}

\begin{frame}[fragile]{Die Lösung}
	\lstinputlisting{resources/02_grundlagen/scope_fix.py}
\end{frame}

\subsection{Objektorientiertes Programmieren}
\begin{frame}[fragile]{Objektorientiertung}
	\begin{itemize}
		\item Python ist auch fundamental objektorientiert
		\item Alles in Python ist ein Objekt
		\item Selbst die Datentypen \texttt{int}, \texttt{bool}, \texttt{str} und \texttt{type} sind Instanzen von \texttt{object} und haben folglich Methoden und Attribute
		\item Der Typ jedes Wertes und jeder Variable lässt sich mit \alert{\texttt{type()}} ermitteln
	\end{itemize}
\end{frame}

\section{Klassen und Attribute}
% TODO: Dieser Frame muss noch mal überarbeitet werden!
\begin{frame}[fragile]{Klassen und Attribute}
	\begin{itemize}
		\item Typen in Python werden ausgedrückt durch Klassen (Keyword \alert{\texttt{class}})
		\item Klassen dienen als Vorlage bzw. Schablone $\rightarrow$ Objekte sind dann Instanzen davon
		\item Die Besonderheit: alle Variablen und Werte sind Instanzen von \texttt{object}
	\end{itemize}
	\texttt{object} und alle Typen selbst sind widerum Objekte, genauer gesagt Instanzen vom Typ \texttt{type} und \texttt{type} widerum ist eine Subklasse von \texttt{object}.
\end{frame}

\begin{frame}[fragile]{Klassen und Attribute}
	\begin{itemize}
		\item Klassen und Objekte können selbst auch Variablen tragen (ähnlich wie in Java)
		\item Man unterscheidet dabei zwischen Klassenattributen und Instanzattributen
	\end{itemize}

	\begin{description}
		\item[Klassenattribute] werden für die Klasse definiert und sind für alle Instanzen gleich
		\item[Instanzattribute] werden außerhalb der Klassendefnition hinzugefügt (normalerweise im Initialisierer) und sind für jede Instanz unterschiedlich.
	\end{description}
\end{frame}

\begin{frame}[fragile]{Klassen und Attribute}
	\begin{itemize}
		\item Zugriff auf Attribute über Punktnotation (wie in Ruby/Java)
		\item Attribute werden wie Variablen mithilfe von \alert{\texttt{=}} gesetzt
		\item Man kann Objekten jederzeit neue Attribute hinzufügen (auch \texttt{type}-Objekten)
		\item Ist außerhalb des Initialisierers nicht empfehlenswert
	\end{itemize}
\end{frame}

\subsection{Klassen- und Objektattribute im Detail}
\begin{frame}[fragile]{Klassen- und Objektattribute im Detail}
	Klassenattribute sind für jede Instanz eines Objektes gleich.
	\lstinputlisting{resources/02_grundlagen/classattributes.py}
\end{frame}

\begin{frame}{Klassen- und Objektattribute im Detail}
	Gewöhnlich definiert man Instanzattribute allerdings im \textbf{Initialisierer}.
	\lstinputlisting{resources/02_grundlagen/instanceattributes.py} 
	\begin{itemize}
		\item Instanzattribute sollten immer in \alert{\texttt{\_\_init\_\_}} definiert werden, um sicherzustellen, dass alle Instanzen die gleichen Attribute haben
	\end{itemize}
\end{frame}

\subsection{Super- und Subklassen}
\begin{frame}[fragile]{Super- und Subklassen}
	Eine Klasse kann Attribute einer anderen Klasse erben, indem sie mit \alert{\texttt{class subclass(superclass):}} definiert wird.
	\begin{itemize}
		\item Subklassen enthalten von Anfang an alle Attribute der Superklasse.
		\item Es können neue Variablen und Methoden hinzugefügt, und auch alte überschrieben werden.
		\item Die Attribute der Superklasse können mit \alert{\texttt{super()}} aufgerufen werden.
	\end{itemize}
\end{frame}

\begin{frame}[fragile]{Super- und Subklassen}
	\lstinputlisting{resources/02_grundlagen/inheritance.py}
\end{frame}


\section{Methoden}
\begin{frame}[fragile]{Methoden}
	\begin{itemize}
		\item Methoden sind Funktionen
		\item Allgemeiner Typ von Methoden ist daher \alert{\texttt{function}}
		\item Methoden liegen im Namespace der zugehörigen Klasse, müssen daher mit \alert{\texttt{ClassName.method\_name()}} angesteuert werden (oder auf dem Objekt aufgerufen werden)
		\item Methoden haben ein implizites erstes Argument (typischerweise \texttt{self} genannt, kann aber variieren)
		\item Beim Aufruf auf einer Instanz wird das Objekt selbst automatisch übergeben
	\end{itemize}
\end{frame}


\subsection{Spezielle Methoden}
\begin{frame}[fragile]{Methoden}
	Dies sind Methoden die auf den meisten Grundlegenden Datenstrukturen implementiert sind, z.B. \texttt{object}.\\
	Die Folgenden beginnen und enden normalerweise mit zwei Unterstrichen.

	\begin{description}
		\item[Initialisierer] Oft auch (fälschlicherweise) Konstruktor genannt.\\ Name: \alert{\texttt{\_\_init\_\_}}\\
			Wird immer aufgerufen wenn eine neue Instanz der Klasse erstellt wird.
		\item[Finalisierer] Oft auch (fälschlicherweise) Destructor genannt.\\
			Name: \alert{\texttt{\_\_del\_\_}}\\
			Wird immer aufgerufen wenn das Objekt vom Garbage Collector aufgeräumt wird. (selten verwendet)
	\end{description}
\end{frame}

\begin{frame}[fragile]{Methoden}
    \begin{description}
		\item[String Konvertierer] Äquivalent zu Java's \texttt{toString} Methode. \\ Name: \alert{\texttt{\_\_str\_\_}}
		\item[String Repräsentation] Ähnlich wie \alert{\texttt{\_\_str\_\_}} aber gedacht für eine für Debug verwendbare Repräsentation anstatt für Output wie \alert{\texttt{\_\_str\_\_}}.
	\end{description}
\end{frame}





\end{document}
