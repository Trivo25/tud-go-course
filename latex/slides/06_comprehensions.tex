% The Slide Definitions
%% Nothing to modify here.
%% make sure to include this before anything else
% \renewcommand*\oldstylenums[1]{{\firaoldstyle #1}}
\documentclass[10pt,aspectratio=169]{beamer}
\usetheme{metropolis}
\usenavigationsymbolstemplate{}
\usepackage{tgcursor}
\usepackage{appendixnumberbeamer}
\usepackage{xcolor}
\usepackage[scale=2]{ccicons}
\usepackage[sfdefault]{FiraSans} %% option 'sfdefault' activates Fira Sans as the default text font
\usepackage[T1]{fontenc}
\usepackage{pgfplots}
\usepgfplotslibrary{dateplot}

\usepackage{xspace}
\newcommand{\themename}{\textbf{\textsc{metropolis}}\xspace}
% packages
\usepackage{color}
\usepackage{listings}
\usepackage{multicol}

% color definitions
\definecolor{mygreen}{rgb}{0,0.6,0}
\definecolor{mygray}{rgb}{0.5,0.5,0.5}
\definecolor{mymauve}{rgb}{0.58,0,0.82}

% re-format the title frame page
\makeatletter
\def\supertitle#1{\gdef\@supertitle{#1}}%
% \setbeamertemplate{title page}
% {
%     \vbox{}
%     \vfill
%     \begin{centering}
%         \begin{beamercolorbox}[sep=8pt,center]{title}
%             \usebeamerfont{supertitle}\@supertitle
%         \end{beamercolorbox}
%         \begin{beamercolorbox}[sep=8pt,center]{title}
%             \usebeamerfont{title}\inserttitle\par%
%             \ifx\insertsubtitle\@empty%
%             \else%
%                 \vskip0.25em%
%                 {\usebeamerfont{subtitle}\usebeamercolor[fg]{subtitle}\insertsubtitle\par}%
%             \fi%
%         \end{beamercolorbox}%
%         \vskip1em\par
%         \begin{beamercolorbox}[sep=8pt,center]{author}
%             \usebeamerfont{author}\insertauthor
%         \end{beamercolorbox}
%         \begin{beamercolorbox}[sep=8pt,center]{institute}
%             \usebeamerfont{institute}\insertinstitute
%         \end{beamercolorbox}
%         \begin{beamercolorbox}[sep=8pt,center]{date}
%             \usebeamerfont{date}\insertdate
%         \end{beamercolorbox}\vskip0.5em
%         {\usebeamercolor[fg]{titlegraphic}\inserttitlegraphic\par}
%     \end{centering}
%     \vfill
% }
\makeatother

% insert frame number
\expandafter\def\expandafter\insertshorttitle\expandafter{%
    \insertshorttitle\hfill%
    \insertframenumber\,/\,\inserttotalframenumber}

% preset-listing options
\lstset{
    backgroundcolor=\color{white},
    % choose the background color;
    % you must add \usepackage{color} or \usepackage{xcolor}
    basicstyle=\footnotesize,
    % the size of the fonts that are used for the code
    breakatwhitespace=false,
    % sets if automatic breaks should only happen at whitespace
    breaklines=true,                 % sets automatic line breaking
    captionpos=b,                    % sets the caption-position to bottom
    commentstyle=\color{mygreen},    % comment style
    % deletekeywords={...},
    % if you want to delete keywords from the given language
    extendedchars=true,
    % lets you use non-ASCII characters;
    % for 8-bits encodings only, does not work with UTF-8
    frame=single,                    % adds a frame around the code
    keepspaces=true,
    % keeps spaces in text,
    % useful for keeping indentation of code
    % (possibly needs columns=flexible)
    keywordstyle=\color{blue},       % keyword style
    % morekeywords={*,...},
    % if you want to add more keywords to the set
    numbers=left,
    % where to put the line-numbers; possible values are (none, left, right)
    numbersep=5pt,
    % how far the line-numbers are from the code
    numberstyle=\tiny\color{mygray},
    % the style that is used for the line-numbers
    rulecolor=\color{black},
    % if not set, the frame-color may be changed on line-breaks
    % within not-black text (e.g. comments (green here))
    stepnumber=1,
    % the step between two line-numbers.
    % If it's 1, each line will be numbered
    stringstyle=\color{mymauve},     % string literal style
    tabsize=4,                       % sets default tabsize to 4 spaces
    title=\lstname
    % show the filename of files included with \lstinputlisting;
    % also try caption instead of title
}

% macro for code inclusion
\newcommand{\includecode}[2][c]{
    \lstinputlisting[caption=#2, style=custom#1]{#2}
}


% Author and Course information
\usepackage[german]{babel}

\usepackage[utf8]{inputenc}

\title{Go}
\subtitle{ Fast, reliable, and efficient software at scale}
\date{\today}
\author{Moritz Schulz, Florian Kluge}
\institute{
  Mail_1\newline
  Florian.Kluge@mailbox.tu-dresden.de
}

\lstset{
  language = Go,
  showspaces = false,
  showtabs = false,
  showstringspaces = false,
  escapechar = @,
  belowskip=-1.5em
}

\def\ContinueLineNumber{\lstset{firstnumber=last}}
\def\StartLineAt#1{\lstset{firstnumber=#1}}
\let\numberLineAt\StartLineAt

\makeatletter


% Presentation title
\title{Comprehensions}
\date{\today}


\begin{document}

\maketitle

\begin{frame}{Gliederung}
	\setbeamertemplate{section in toc}[sections numbered]
	\tableofcontents
\end{frame}


\section{Basics}
\begin{frame}{Basics}
  Comprehensions sind eine bequeme Art und Weise, um Funktoren (Datenstrukturen, die andere Datenstrukturen beinhalten) mit kleinen Expressions zu erstellen und zu füllen und sind in allen modernen Sprachen vorhanden.
\end{frame}



\section{List Comprehension}
\begin{frame}{List Comprehension}
  Grundlegender Syntax: \alert{\texttt{[ EXPRESSION for LAUFVARIABLE in ITERABLE (if FILTER) ]}}\\
  \begin{description}
    \item[EXPRESSION] Ist ein beliebiger Ausdruck (man stelle sich ein implizites \texttt{return} vor), etwa ein Wert, eine Variable, eine Gleichung, etc ... \\
    	\texttt{EXPRESSION} wird am Ende in der Liste abgelegt.
    \item[LAUFVARIABLE] Eine beliebige Variable, die in \textit{EXPRESSION} und \textit{FILTER} zur Verfügung steht
    \item[ITERABLE] Ist häufig etwas wie \texttt{range()} oder eine andere Liste.
    \item[FILTER] Eine optionale boolean expression, womit Einträge gefiltert werden (falls \texttt{False}). N\"utzlich, wenn z.B. nur gerade Zahlen \"ubernommen werden sollen, usw...
  \end{description}

\end{frame}

\begin{frame}{List Comprehension - Beispiel}
	\lstinputlisting{resources/06_comprehensions/lists.py}
\end{frame}



\section{Dict Comprehension}
\begin{frame}{Dict Comprehension}
  Grundlegender Syntax: \alert{\texttt{\{ KEY : VALUE for LAUFVARIABLE in ITERABLE (if FILTER) \}}}\\[.75cm]
  Fast der gleiche Syntax, nur dieses Mal mit 2 Expressions: \textit{KEY} und \textit{VALUE}. Ansonsten gelten die gleichen Regeln.
\end{frame}

\begin{frame}{Dict Comprehension - Beispiel}
	\lstinputlisting{resources/06_comprehensions/dicts.py}
\end{frame}



\section{Generators}
\begin{frame}{Generators}
	\begin{description}
		\item[Generator] Ein Objekt, \"uber das iteriert werden kann. Wenn ein Element daraus verwendet wurde, ist es nicht mehr in dem Generatorobjekt enthalten.\\[.5cm]
	\end{description}
	
	Die grundlegende Syntax ist gleich der einer \textit{List Comprehension}.
	Da sich \alert{\texttt{list}} und \alert{\texttt{dict}} auch aus Iterables bauen lassen, gilt prinzipiell:\\[.25cm]

		\alert{\texttt{list(EXPRESSION for VARIABLE in ITERABLE) == [EXPRESSION for VARIABLE in ITERABLE]}}\\[.25cm]
		und\\[.25cm]
		\alert{\texttt{dict((KEY, VALUE) for VARIABLE in ITERABLE) == \{KEY:VALUE for VARIABLE in ITERABLE\}}}\\[.25cm]

	\textbf{Aber:} Generators verhalten sich anders als Lists oder Dicts!

\end{frame}


%\section{Misc}
%\begin{frame}{Misc}
%  Mit \texttt{tuple()} und der generator expression lassen sich theoretisch auch tuple comprehensions bauen, ist aber konzeptionell eher Unsinn.
%\end{frame}


% nothing to do from here on
\end{document}
