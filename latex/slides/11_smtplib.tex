% The Slide Definitions
%% Nothing to modify here.
%% make sure to include this before anything else
% \renewcommand*\oldstylenums[1]{{\firaoldstyle #1}}
\documentclass[10pt,aspectratio=169]{beamer}
\usetheme{metropolis}
\usenavigationsymbolstemplate{}
\usepackage{tgcursor}
\usepackage{appendixnumberbeamer}
\usepackage{xcolor}
\usepackage[scale=2]{ccicons}
\usepackage[sfdefault]{FiraSans} %% option 'sfdefault' activates Fira Sans as the default text font
\usepackage[T1]{fontenc}
\usepackage{pgfplots}
\usepgfplotslibrary{dateplot}

\usepackage{xspace}
\newcommand{\themename}{\textbf{\textsc{metropolis}}\xspace}
% packages
\usepackage{color}
\usepackage{listings}
\usepackage{multicol}

% color definitions
\definecolor{mygreen}{rgb}{0,0.6,0}
\definecolor{mygray}{rgb}{0.5,0.5,0.5}
\definecolor{mymauve}{rgb}{0.58,0,0.82}

% re-format the title frame page
\makeatletter
\def\supertitle#1{\gdef\@supertitle{#1}}%
% \setbeamertemplate{title page}
% {
%     \vbox{}
%     \vfill
%     \begin{centering}
%         \begin{beamercolorbox}[sep=8pt,center]{title}
%             \usebeamerfont{supertitle}\@supertitle
%         \end{beamercolorbox}
%         \begin{beamercolorbox}[sep=8pt,center]{title}
%             \usebeamerfont{title}\inserttitle\par%
%             \ifx\insertsubtitle\@empty%
%             \else%
%                 \vskip0.25em%
%                 {\usebeamerfont{subtitle}\usebeamercolor[fg]{subtitle}\insertsubtitle\par}%
%             \fi%
%         \end{beamercolorbox}%
%         \vskip1em\par
%         \begin{beamercolorbox}[sep=8pt,center]{author}
%             \usebeamerfont{author}\insertauthor
%         \end{beamercolorbox}
%         \begin{beamercolorbox}[sep=8pt,center]{institute}
%             \usebeamerfont{institute}\insertinstitute
%         \end{beamercolorbox}
%         \begin{beamercolorbox}[sep=8pt,center]{date}
%             \usebeamerfont{date}\insertdate
%         \end{beamercolorbox}\vskip0.5em
%         {\usebeamercolor[fg]{titlegraphic}\inserttitlegraphic\par}
%     \end{centering}
%     \vfill
% }
\makeatother

% insert frame number
\expandafter\def\expandafter\insertshorttitle\expandafter{%
    \insertshorttitle\hfill%
    \insertframenumber\,/\,\inserttotalframenumber}

% preset-listing options
\lstset{
    backgroundcolor=\color{white},
    % choose the background color;
    % you must add \usepackage{color} or \usepackage{xcolor}
    basicstyle=\footnotesize,
    % the size of the fonts that are used for the code
    breakatwhitespace=false,
    % sets if automatic breaks should only happen at whitespace
    breaklines=true,                 % sets automatic line breaking
    captionpos=b,                    % sets the caption-position to bottom
    commentstyle=\color{mygreen},    % comment style
    % deletekeywords={...},
    % if you want to delete keywords from the given language
    extendedchars=true,
    % lets you use non-ASCII characters;
    % for 8-bits encodings only, does not work with UTF-8
    frame=single,                    % adds a frame around the code
    keepspaces=true,
    % keeps spaces in text,
    % useful for keeping indentation of code
    % (possibly needs columns=flexible)
    keywordstyle=\color{blue},       % keyword style
    % morekeywords={*,...},
    % if you want to add more keywords to the set
    numbers=left,
    % where to put the line-numbers; possible values are (none, left, right)
    numbersep=5pt,
    % how far the line-numbers are from the code
    numberstyle=\tiny\color{mygray},
    % the style that is used for the line-numbers
    rulecolor=\color{black},
    % if not set, the frame-color may be changed on line-breaks
    % within not-black text (e.g. comments (green here))
    stepnumber=1,
    % the step between two line-numbers.
    % If it's 1, each line will be numbered
    stringstyle=\color{mymauve},     % string literal style
    tabsize=4,                       % sets default tabsize to 4 spaces
    title=\lstname
    % show the filename of files included with \lstinputlisting;
    % also try caption instead of title
}

% macro for code inclusion
\newcommand{\includecode}[2][c]{
    \lstinputlisting[caption=#2, style=custom#1]{#2}
}


% Author and Course information
\usepackage[german]{babel}

\usepackage[utf8]{inputenc}

\title{Go}
\subtitle{ Fast, reliable, and efficient software at scale}
\date{\today}
\author{Moritz Schulz, Florian Kluge}
\institute{
  Mail_1\newline
  Florian.Kluge@mailbox.tu-dresden.de
}

\lstset{
  language = Go,
  showspaces = false,
  showtabs = false,
  showstringspaces = false,
  escapechar = @,
  belowskip=-1.5em
}

\def\ContinueLineNumber{\lstset{firstnumber=last}}
\def\StartLineAt#1{\lstset{firstnumber=#1}}
\let\numberLineAt\StartLineAt

\makeatletter


% Presentation title
% TODO Change the topic of the lesson
\title{Mails in Python senden}
\date{\today}


\begin{document}

\maketitle

\begin{frame}{Gliederung}
	\setbeamertemplate{section in toc}[sections numbered]
	\tableofcontents
\end{frame}

\begin{frame}
	Die folgenden Folien enthalten eine praktische Anleitung zum Senden von Mails in Python.
\end{frame}


\section{Grundlagen: Mails senden}
\subsection{Das Modul \texttt{smtplib}}
\begin{frame}[fragile]{Das Modul \texttt{smtplib}}
	Das Modul \texttt{\alert{smtplib}} definiert eine SMTP\textbf{*}-Client Session, die genutzt werden kann, um von jedem beliebigen, internetf\"ahigen Ger\"at E-Mails zu verschicken. \\[1cm]
	
	\textbf{*}\textit{SMTP} steht für \textbf{S}imple \textbf{M}ail \textbf{T}ransfer 
	\textbf{P}rotocol und ist das Standard-Protokoll zum E-Mail Versand.
\end{frame}

\begin{frame}[fragile]{Verbindung zum Server}
	Die smtplib kann sich zu einem SMTP-Server verbinden\\[.5cm]
	\lstinputlisting[firstline=3, lastline=3]{resources/11_sendmail/defs.py}
	
	\ \\[.25cm]
	Der Server kann zum einen als einheitlicher \texttt{String} (inkl. Port) oder einzeln als \texttt{host} und \texttt{port} angeben werden.
\end{frame}

\begin{frame}[fragile]{Login auf dem Server}
	Heutzutage arbeiten die meisten SMTP-Server mit \texttt{TLS}, um eine sichere Verbindung zu gewährleisten. Diese muss mithilfe der Methode \texttt{\alert{starttls()}} hergestellt werden.\\[.25cm]
	Der eigentliche Login erfolgt im Anschluss durch:\\[.25cm]
	\lstinputlisting[firstline=5, lastline=5]{resources/11_sendmail/defs.py}
	\ \\[.25cm]
	Der Parameter \texttt{initial\_response\_ok} kann in unserem Fall vernachlässigt werden.
\end{frame}

\begin{frame}[fragile]{Senden der Mail}
	Das tatsächliche Versenden der Mail funktioniert dann mit folgender Methode:\\[.25cm]
	\lstinputlisting[firstline=7]{resources/11_sendmail/defs.py}
	\ \\[.25cm]
	Jedoch l\"asst sich das ganze auch mit einem \texttt{MIME}-Objekt vereinfachen, auf das später noch eingegangen wird. \\
	Dieses wird mit \texttt{\alert{send\_message()}} versendet.
\end{frame}

\begin{frame}{Schließen der Verbindung}
	Zum Schluss darf nicht vergessen werden, die Verbindung zum SMTP-Server wieder zu schließen.\\
	Dies geschieht mit der Methode \texttt{\alert{quit()}} oder man stellt die Verbindung mithilfe eines \textit{Filehandlers} her.
\end{frame}

\begin{frame}[fragile]{Die vollst\"andige Serverkommunikation}
	\lstinputlisting[linerange={3-3,19-34}]{resources/11_sendmail/mailer.py}
	%\ \\[1cm]
	Auf \texttt{buildmessage()} wird im folgenden Teil eingegangen.
\end{frame}

\section{Komplexere Mails senden}
\subsection{Das Modul \texttt{email}}
\begin{frame}[fragile]{Das Modul \texttt{email}}
	Um mehr Möglichkeiten zur Gestaltung der E-Mail zu haben, lohnt sich die Verwendung des Moduls \texttt{email}.\\[.5cm]
	Mit \texttt{\alert{email.mime}} lassen sich Emails individuell bauen und zusammensetzen. Außerdem kann man mehrteilige Mails und Mails mit Anhängen (z.B. Bildern) erstellen.
\end{frame}

\begin{frame}[fragile]{Die Klasse \texttt{MIMEText}}
	Die Klasse \texttt{email.mime.text.MIMEText()} erstellt ein MIME Objekt, welches hautps\"achlich aus Text besteht und einfach dem SMTP-Objekt \"ubergeben werden kann: \\ \ \\
	\lstinputlisting[lastline=1]{resources/11_sendmail/defs.py} \ \\
	
	\begin{description}
		\item[\_text] Ein String, der den Inhalt der Nachricht enth\"alt.
		\item[\_subtype] Der Untertyp des Objekts, per default \texttt{plain}
		\item[charset] Der Zeichensatz, der zur Kodierung der Zeichen verwendet werden soll. Standardm\"a\ss{}ig \texttt{us-ascii} oder \texttt{utf8}, abh\"angig von dem eingegebenen Text.
	\end{description}
\end{frame}

\begin{frame}[fragile]{Die Meta-Felder}
	Wenn das \texttt{MIMEText} Objekt instanziiert ist, muss dieses mit weiteren Informationen erg\"anzt werden:
	\lstinputlisting[firstline=10,lastline=15]{resources/11_sendmail/mailer.py}
	\begin{description}
		\item[From] Daten des Absenders (nur Mail oder Name und Mail)
		\item[To] Daten des Empfängers (nur Mail oder Name und Mail)
		\item[Subject] Betreff der Mail
	\end{description}
	Die Message l\"asst sich au\ss{}erdem noch um einen \texttt{Cc} oder einen \texttt{Bcc} erweitern.
\end{frame}

\begin{frame}{Das fertige Mail-Skript}
	\lstinputlisting[firstline=4,lastline=17]{resources/11_sendmail/mailer.py}
\end{frame}

% nothing to do from here on
\end{document}
